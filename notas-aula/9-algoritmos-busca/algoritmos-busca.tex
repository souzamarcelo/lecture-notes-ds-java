\newcommand{\templatesdir}{../../../templates}
\newcommand{\template}{template-roteiro-est}
\input{\templatesdir/\template/template}

\newcommand{\content}{Busca em estruturas lineares}
\newcommand{\class}{Algoritmos e Estruturas de Dados}
\newcommand{\shortcourse}{45EST}

\begin{document}

\makeheader

{
Leitura obrigatória:
\begin{itemize}
	\item Capítulo 5 de~\cite{Ziviani2010} -- Pesquisa em memória binária.
\end{itemize}

Leitura complementar:
\begin{itemize}
	\item Capítulo 13 de~\cite{Pereira2008} -- Ordenação e busca.
\end{itemize}
}

\medskip

\newtitle{Algoritmos de busca}

\begin{itemize}
	\item Buscar um elemento consiste em verificar se o mesmo está armazenado em uma estrutura de dados.
	\item O retorno pode ser o próprio elemento, o índice onde ele se encontra, ou um valor booleano de sucesso.

	{\color{redtext}
	\item Busca em um vetor simples:
	\begin{itemize}
		\item Dado um vetor e um valor, verifica se o valor está armazenado no vetor.
	\end{itemize}
	\item Busca em uma lista (vetor ou encadeada) genérica:
	\begin{itemize}
		\item Dada uma lista genérica e um valor genérico, verifica se o valor está armazenado na lista.
	\end{itemize}
	\item Busca em uma estrutura de entradas (chave, valor):
	\begin{itemize}
		\item Dada uma lista genérica de entradas e uma chave, devolve o valor correspondente, ou \texttt{null} caso a chave não seja encontrada.
	\end{itemize}
	}
\end{itemize}

\clearpage

\newtitle{Busca sequencial}

\begin{itemize}
	\item É a forma mais simples de busca.
	\item Percorre a estrutura até encontrar o elemento.
	\item Complexidade linear $O(n)$.
\end{itemize}

\medskip

Busca sequencial simples:
\begin{minted}{java}
public class SimpleSequentialSearch {
	
	public int search(int[] array, int value) {
		for(int i = 0; i < array.length; i++)
		if(array[i] == value)
			return i;
		return -1;
	}
	
	public int search(String[] array, String value) {
		for(int i = 0; i < array.length; i++)
		if(array[i].equals(value))
			return i;	
		return -1;
	}
}
\end{minted}

\medskip

Busca sequencial genérica:
\begin{minted}{java}
public class GenericSequentialSearch<E> {
	
	Comparator<E> comp;
	
	public GenericSequentialSearch() {
		this(new DefaultComparator<E>());
	}
	
	public GenericSequentialSearch(Comparator<E> c) {
		comp = c;
	}
	
	public int indexOf(E[] array, E value) {
		for(int i = 0; i < array.length; i++)
		if(comp.compare(array[i], value) == 0)
			return i;
		return -1;
	}
	
	public int indexOf(List<E> array, E value) {
		for(int i = 0; i < array.size(); i++)
		if(comp.compare(array.get(i), value) == 0)
			return i;
		return -1;
	}
}
\end{minted}

\medskip

{\color{redtext}
Comentários:
\begin{itemize}
	\item Classe usa um comparador para identificar o elemento buscado.
	\begin{enumerate}
		\item Definir um comparador próprio; ou
		\item Definir a classe comparável e implementar o método \texttt{compareTo}.
	\end{enumerate}
\end{itemize}
}

\medskip

Busca sequencial genérica para entradas:
\begin{minted}{java}
public class GenericSequentialEntrySearch<K, V> {
	
	Comparator<K> comp;
	
	public GenericSequentialEntrySearch() {
		this(new DefaultComparator<K>());
	}
	
	public GenericSequentialEntrySearch(Comparator<K> c) {
		comp = c;
	}
	
	public V searchElement(List<Entry<K,V>> array, K key) {
		for(int i = 0; i < array.size(); i++)
			if(comp.compare(array.get(i).getKey(), key) == 0)
				return array.get(i).getValue();
		return null;
	}
}
\end{minted}

\medskip

{\color{redtext}
	Comentários:
	\begin{itemize}
		\item Novamente, a classe usa um comparador para encontrar a chave.
	\end{itemize}
}

\medskip

\newtitle{Busca binária}

\begin{itemize}
	\item Busca mais eficaz, executada em $O(\log n)$.
	\item Pré-condições:
	\begin{itemize}
		\item Necessita acesso aleatório à estrutura (vetores).
		\item Dados devem estar ordenados.
	\end{itemize}
	\item Funcionamento:
	\begin{enumerate}
		\item Avalia o elemento central da lista.
		\item Caso seja o elemento buscado -- sucesso.
		\item Caso contrário, avalia em qual sub-lista se o elemento pode estar.
		\item Repete a busca com a sub-lista correspondente.
	\end{enumerate}
\end{itemize}

\clearpage

\textbf{Exemplo}

Vetor (já ordenado):
\vspace{-10pt}

\begin{figure}[H]
	\centering
	\includegraphics[width=0.7\linewidth]{img/figure-5-4}
\end{figure}

Elemento buscado: $22$.

\medskip

Funcionamento:

\begin{figure}[H]
	\centering
	\includegraphics[width=0.7\linewidth]{img/figure-5-5}
\end{figure}

Conclusão:
\begin{itemize}
	\item Encontra o elemento com 4 avaliações ($14$, $25$, $19$ e $22$).
	\item Caso o elemento buscado fosse $23$, na próxima iteração identificaria sua inexistência, pois \texttt{high < low}.
\end{itemize}

\clearpage

Busca binária simples:
\begin{minted}{java}
public class SimpleBinarySearch {
	public int search(int[] array, int value) {
		int start = 0;
		int end = array.length - 1;
		int mid;
		
		do {
			mid = (start + end) / 2;
			if(array[mid] < value)
				start = mid + 1;
			else
				end = mid - 1;	
		} while(array[mid] != value && start <= end);
		
		if(array[mid] == value)
			return mid;
		else
			return -1;
	}
	
	public int search(String[] array, String value) {
		int start = 0;
		int end = array.length - 1;
		int mid;
		
		do {
			mid = (start + end) / 2;
			if(array[mid].compareTo(value) < 0)
				start = mid + 1;
			else
				end = mid - 1;
		} while(array[mid].compareTo(value) != 0 && start <= end);
		
		if(array[mid].compareTo(value) == 0)
			return mid;
		else
			return -1;
	}
}
\end{minted}

\medskip

{\color{redtext}
Comentários:
\begin{itemize}
	\item Veriáveis \texttt{start} e \texttt{end} delimitam a sub-lista de busca (equivalente a \texttt{low} e \texttt{high}).
	\item Enquanto não encontra, atualiza o intervalo de índices e repete o processo.
	\item Quando \texttt{start > end}, finaliza a busca sem sucesso.
\end{itemize}
}

\medskip

Busca binária genérica:
\begin{minted}{java}
public class GenericBinarySearch<E> {
	
	Comparator<E> comp;
	
	public GenericBinarySearch() {
		this(new DefaultComparator<E>());
	}
	
	public GenericBinarySearch(Comparator<E> c) {
		comp = c;
	}
	
	public int indexOf(E[] array, E value) {
		int start = 0;
		int end = array.length - 1;
		int mid;
		
		do {
			mid = (start + end) / 2;
			if(comp.compare(array[mid], value) < 0)
				start = mid + 1;
			else
				end = mid - 1;		
		} while(comp.compare(array[mid], value) != 0 && start <= end);
		
		if(comp.compare(array[mid], value) == 0)
			return mid;
		else
			return -1;
	}
	
	public int indexOf(List<E> array, E value) {
		int start = 0;
		int end = array.size() - 1;
		int mid;
		
		do {
			mid = (start + end) / 2;
			if(comp.compare(array.get(mid), value) < 0)
				start = mid + 1;
			else
				end = mid - 1;	
		} while(comp.compare(array.get(mid), value) != 0 && start <= end);
		
		if(comp.compare(array.get(mid), value) == 0)
			return mid;
		else
			return -1;
	}
}
\end{minted}

\medskip

Busca binária genérica para entradas:
\begin{minted}{java}
public class GenericBinaryEntrySearch<K,V> {
	
	Comparator<K> comp;
	
	public GenericBinaryEntrySearch() {
		this(new DefaultComparator<K>());
	}
	
	public GenericBinaryEntrySearch(Comparator<K> c) {
		comp = c;
	}
	
	public V searchElement(List<Entry<K,V>> array, K key) {
		int start = 0;
		int end = array.size() - 1;
		int mid;
		
		do {
			mid = (start + end) / 2;
			if(comp.compare(array.get(mid).getKey(), key) < 0)
				start = mid + 1;
			else
				end = mid - 1;		
		} while(comp.compare(array.get(mid).getKey(), key) != 0 && start <= end);
		
		if(comp.compare(array.get(mid).getKey(), key) == 0)
			return array.get(mid).getValue();
		else
			return null;
	}
}
\end{minted}

\medskip

{\color{redtext}
	Comentários:
	\begin{itemize}
		\item Veja as aplicações das buscas em diferentes estruturas nas classes \texttt{TestSequentialSearch} e \texttt{TestBinarySearch}.
	\end{itemize}
}

\clearpage

\newtitle{Atividades}

\begin{enumerate}
	\item Desenvolva um software para gerenciar as contas de um banco. Primeiro, armazene os dados em uma estrutura de dados não-ordenada e utilize o algoritmo de busca sequencial para buscar contas. Após isso, implemente uma versão utilizando uma estrutura de dados ordenada, e aplique o algoritmo de busca binária para a pesquisa. Compare o tempo de execução de ambas abordagens para diferentes quantidades de contas armazenadas. Verifique a complexidade empírica dos algoritmos e compare com a complexidade no pior caso.

	\item Implemente uma versão recursiva do algoritmo de busca binária. Em caso de dúvidas, consulte a Seção 5.1.3 de~\cite{GoodrichEtAl2014}.
	
	\item Simule o algoritmo de busca binária para os seguintes casos:
	\begin{enumerate}[a)]
		\item \texttt{x = 15}, \texttt{v = \{15, 27, 33, 46, 51, 63, 71, 82, 90\}}.
		\item \texttt{x = 33}, \texttt{v = \{15, 27, 33, 46, 51, 63, 71, 82, 90\}}.
		\item \texttt{x = 63}, \texttt{v = \{15, 27, 33, 46, 51, 63, 71, 82, 90\}}.
		\item \texttt{x = 81}, \texttt{v = \{15, 27, 33, 46, 51, 63, 71, 82, 90\}}.
		\item \texttt{x = 22}, \texttt{v = \{15, 27, 33, 46, 51, 63, 71, 82, 90\}}.
	\end{enumerate}
	Compare o número de avaliações realizadas com o número de avaliações que uma busca sequencial faria.
	
	\item Quando o vetor está ordenado, a busca sequencial não precisa percorrer toda a lista para saber que o elemento buscado não existe. Ela pode parar quando o elemento analisado for maior que o buscado. Implemente as modificações necessárias para esta estratégia. Qual a complexidade no pior caso do novo algoritmo?
	
	\item Implemente os algoritmos de busca sequencial e binária nas estruturas de dados estudadas em sala de aula.
	
	\item Veja as demonstrações de buscas sequencial e binária disponíveis em \url{https://www.cs.usfca.edu/~galles/visualization/Search.html}. 
\end{enumerate}

\medskip

\newtitle{Referências}
\begingroup
	\footnotesize
	\renewcommand{\chapter}[2]{}%
	\bibliographystyle{apalike}
	\bibliography{../referencias}
\endgroup

\end{document}